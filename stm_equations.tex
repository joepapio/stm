%%%%%%%%%%%%%%%%%%%%%%%%%%%%%%%%%%%%%%%%%
% Short Sectioned Assignment
% LaTeX Template
% Version 1.0 (5/5/12)
%
% This template has been downloaded from:
% http://www.LaTeXTemplates.com
%
% Original author:
% Frits Wenneker (http://www.howtotex.com)
%
% License:
% CC BY-NC-SA 3.0 (http://creativecommons.org/licenses/by-nc-sa/3.0/)
%
%%%%%%%%%%%%%%%%%%%%%%%%%%%%%%%%%%%%%%%%%

%----------------------------------------------------------------------------------------
%	PACKAGES AND OTHER DOCUMENT CONFIGURATIONS
%----------------------------------------------------------------------------------------

\documentclass[paper=letter, fontsize=12pt]{scrartcl} %  and 11pt font size

\usepackage[T1]{fontenc} % Use 8-bit encoding that has 256 glyphs
%\usepackage{fourier} % Use the Adobe Utopia font for the document - comment this line to return to the LaTeX default
\usepackage[english]{babel} % English language/hyphenation
\usepackage{amsmath,amsfonts,amsthm} % Math packages

\usepackage{float}
\usepackage{placeins}
\usepackage{bm}

\usepackage{subcaption}

\usepackage{graphicx}
\usepackage{enumitem}
\usepackage{graphicx}

\usepackage{lipsum} % Used for inserting dummy 'Lorem ipsum' text into the template

\usepackage{sectsty} % Allows customizing section commands
\allsectionsfont{\centering \normalfont\scshape} % Make all sections centered, the default font and small caps

\usepackage{fancyhdr} % Custom headers and footers
\pagestyle{fancyplain} % Makes all pages in the document conform to the custom headers and footers
\fancyhead{} % No page header - if you want one, create it in the same way as the footers below
\fancyfoot[L]{} % Empty left footer
\fancyfoot[C]{} % Empty center footer
\fancyfoot[R]{\thepage} % Page numbering for right footer
\renewcommand{\headrulewidth}{0pt} % Remove header underlines
\renewcommand{\footrulewidth}{0pt} % Remove footer underlines
\setlength{\headheight}{13.6pt} % Customize the height of the header

\numberwithin{equation}{section} % Number equations within sections (i.e. 1.1, 1.2, 2.1, 2.2 instead of 1, 2, 3, 4)
\numberwithin{figure}{section} % Number figures within sections (i.e. 1.1, 1.2, 2.1, 2.2 instead of 1, 2, 3, 4)
\numberwithin{table}{section} % Number tables within sections (i.e. 1.1, 1.2, 2.1, 2.2 instead of 1, 2, 3, 4)

\setlength\parindent{0pt} % Removes all indentation from paragraphs - comment this line for an assignment with lots of text

%----------------------------------------------------------------------------------------
%	TITLE SECTION
%----------------------------------------------------------------------------------------

\newcommand{\horrule}[1]{\rule{\linewidth}{#1}} % Create horizontal rule command with 1 argument of height

\title{	
\normalfont \normalsize 
\textsc{Iowa State University} \\ [25pt] % Your university, school and/or department name(s)
\horrule{0.5pt} \\[0.4cm] % Thin top horizontal rule
\huge stm euqations \\ % The assignment title
\horrule{2pt} \\[0.5cm] % Thick bottom horizontal rule
}

\author{Joe Papio} % Your name



\date{\normalsize Date} % Today's date or a custom date - use \today for today's date

\begin{document}

\maketitle % Print the title

%----------------------------------------------------------------------------------------
%	PROBLEM 1
%----------------------------------------------------------------------------------------

\section{Introduction}






\section{Model}

\centering Topic Prevalence:
\begin{align*}
	\mu_{d,k} &= X_d\gamma_k \\
	\gamma_k &\sim \mathcal{N}(0,\sigma_k^2)\\
	\sigma_k^2 &\sim Gamma(s^{\gamma}, r^{\gamma})
\end{align*}

\centering Language Model:
\begin{align*}
\theta_{d} &= LogisticNormal(\mu_d, \Sigma) \\
z_{d,n} &\sim Mult(\theta_d)\\
w_{d,n} &\sim Mult(\beta^{k=z_d,n}_d, r^{\gamma})
\end{align*}

\centering Topical Content:
\begin{align*}
\beta_{d,v}^k &\propto exp(m_v+\kappa_v^{.,k}+\kappa_v^{y,.}+\kappa_v^{y,k}) \\
\kappa_v^{y,k} &\sim Laplace(0,\tau_v^{y,k})\\
\tau_v^{y,k} &\sim Gamma(s^{\kappa}, r^{\kappa})
\end{align*}


\begin{equation*}
\theta_i \stackrel{ind}{\sim} Beta(\alpha, \beta).
\end{equation*}

	\begin{align*}
	 \alpha &=\eta*\mu \\ 
	\beta &= \eta*(1-\mu) 
%	\mu &\sim Beta(1,1) \\
%	\eta & \sim LogNormal(0,1)
	\end{align*}

	\begin{align*}
%	p(\mu, \eta) & \propto p() \\
	p(\theta_i) & \propto p(\mu)p(\eta) \\
	p(\mu) & \propto Beta(1,1) \\
	p(\eta) & \propto LNorm(0,1).
	\end{align*}



\begin{equation*}
p(\theta_1...\theta_{51},\mu,\eta|\mathbf{y}) \propto \prod_{n=1}^{51}\Big[p(y_i|\theta_i)p(\theta_i|\mu,\eta)\Big] p(\mu)p(\eta).
\end{equation*}

%------------------------------------------------

\section{Results}


\begin{equation}
\bigg(\prod^D_{d=1}N(\mathbf{\eta_d}|\mathbf{X}_d\mathbf{\gamma}, \mathbf{\Sigma}) \Big(\prod^N_{n=1}Mult(z_{n,d}|\mathbf{\theta}_d) \times Mult(w_n|\mathbf{\beta}_{d,k=z_{d,n}} \Big) \bigg) \times \Pi p(\kappa) \Pi p(\mathbf{\Gamma})
\end{equation}

\end{document}


"where $\Gamma = [\gamma_1|...|\gamma_K] $ is a P x (K-1) matrix of topic prev coeffs for topic prev model specified by (1) and (2), {$\kappa^{(t)}_{.,.},\kappa^{(c)}_{.,.},\kappa^{(i)}_{.,.}$} is a collection of coefficients for the topical content model specified by (5)"

core language model is denoted by (3) and (4) "allowsfor correlations in the topic proportions using the logistic normal"


"represent the logistic normal by drawing $\mathbf{\eta}_d \sim Normal_{K-1}(\mathbf{\mu}_d, \mathbf{\Sigma})$ and mapping to the simplex by specifying 

"$\theta_{d,k} = exp(\eta_{d,k}/\Sigma^K_{i=1}exp(\eta_{d,i})) $ where $\eta_{d,K} $ is fixed to zero for identifiabiity."

given $\mathbf{\theta}_d$, for each word in document $d$, topic is sampled from (3), and then conditional on that, word selected over terms $\mathbf{\beta_{z_{d,n}}}$

"model mean vector of logistic normal as simple linear model such that $\mathbf{\mu_d = \Gamma'}\mathbf{x}'_d$ w/ additional regularizing prior on elements of $\mathbf{\Gamma}$ to avoid over-fitting"

"parameterize multinomial dist of word occurances in terms of log-transformed rate deviations from the rates of corpus-wide background distribution, \textbf{m}

"In the proposed model, the log-transformed rate deviations are denoted by a collection of parameters { $\kappa$} where the superscript indicates which set they belong to, that is, $t$ for topics, $c$ for covariates, or $i$ for topic-covariate interactions. In detail, $\kappa^{(t)}$ is a K by V matrix containing the log-transformed rate deviations for each topic $k$ and term $v$, over the baseline log-transformed rate for term $v$. These deviations are shared across all A levels of the content covariate $Y_d$. The matrix  $\kappa^{(c)}$ has dimension A × V, and it contains the log- transformed rate deviation for each level of the covariate $Y_d$ and each term $v$, over the baseline log-transformed rate for term $v$ . These deviations are shared across all topics. Finally, the array $\kappa^{(i)}$ has dimension A × K × V , and it collects the covariate-topic interaction effects. For example, for the simple case where there is a single covariate ($Y_d$) denoting a mutually exclusive and exhaustive group of documents, such as newswire source, the distribution over terms is obtained by adding these log- transformed effects such that the rate $\beta_{d,k,v} \propto exp(m_v + \kappa^{(t)}_{k,v} + \kappa^{(c)}_{y_d,v} \kappa^{(i)}_{y_d,k,v } )$

