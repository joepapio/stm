%%%%%%%%%%%%%%%%%%%%%%%%%%%%%%%%%%%%%%%%%
% Short Sectioned Assignment
% LaTeX Template
% Version 1.0 (5/5/12)
%
% This template has been downloaded from:
% http://www.LaTeXTemplates.com
%
% Original author:
% Frits Wenneker (http://www.howtotex.com)
%
% License:
% CC BY-NC-SA 3.0 (http://creativecommons.org/licenses/by-nc-sa/3.0/)
%
%%%%%%%%%%%%%%%%%%%%%%%%%%%%%%%%%%%%%%%%%

%----------------------------------------------------------------------------------------
%	PACKAGES AND OTHER DOCUMENT CONFIGURATIONS
%----------------------------------------------------------------------------------------

\documentclass[paper=letter, fontsize=12pt]{scrartcl} %  and 11pt font size

\usepackage[T1]{fontenc} % Use 8-bit encoding that has 256 glyphs
%\usepackage{fourier} % Use the Adobe Utopia font for the document - comment this line to return to the LaTeX default
\usepackage[english]{babel} % English language/hyphenation
\usepackage{amsmath,amsfonts,amsthm} % Math packages

\usepackage{float}
\usepackage{placeins}
\usepackage{bm}

\usepackage{subcaption}

\usepackage{graphicx}
\usepackage{enumitem}
\usepackage{graphicx}

\usepackage{lipsum} % Used for inserting dummy 'Lorem ipsum' text into the template

\usepackage{sectsty} % Allows customizing section commands
\allsectionsfont{\centering \normalfont\scshape} % Make all sections centered, the default font and small caps

\usepackage{fancyhdr} % Custom headers and footers
\pagestyle{fancyplain} % Makes all pages in the document conform to the custom headers and footers
\fancyhead{} % No page header - if you want one, create it in the same way as the footers below
\fancyfoot[L]{} % Empty left footer
\fancyfoot[C]{} % Empty center footer
\fancyfoot[R]{\thepage} % Page numbering for right footer
\renewcommand{\headrulewidth}{0pt} % Remove header underlines
\renewcommand{\footrulewidth}{0pt} % Remove footer underlines
\setlength{\headheight}{13.6pt} % Customize the height of the header

\numberwithin{equation}{section} % Number equations within sections (i.e. 1.1, 1.2, 2.1, 2.2 instead of 1, 2, 3, 4)
\numberwithin{figure}{section} % Number figures within sections (i.e. 1.1, 1.2, 2.1, 2.2 instead of 1, 2, 3, 4)
\numberwithin{table}{section} % Number tables within sections (i.e. 1.1, 1.2, 2.1, 2.2 instead of 1, 2, 3, 4)

\setlength\parindent{0pt} % Removes all indentation from paragraphs - comment this line for an assignment with lots of text

%----------------------------------------------------------------------------------------
%	TITLE SECTION
%----------------------------------------------------------------------------------------

\newcommand{\horrule}[1]{\rule{\linewidth}{#1}} % Create horizontal rule command with 1 argument of height

\title{	
\normalfont \normalsize 
\textsc{Iowa State University} \\ [25pt] % Your university, school and/or department name(s)
\horrule{0.5pt} \\[0.4cm] % Thin top horizontal rule
\huge stm euqations \\ % The assignment title
\horrule{2pt} \\[0.5cm] % Thick bottom horizontal rule
}

\author{Joe Papio} % Your name



\date{\normalsize Date} % Today's date or a custom date - use \today for today's date

\begin{document}

\maketitle % Print the title

%----------------------------------------------------------------------------------------
%	PROBLEM 1
%----------------------------------------------------------------------------------------

\section{Introduction}






\section{Model}

\centering Topic Prevalence:
\begin{align*}
	\mu_{d,k} &= X_d\gamma_k \\
	\gamma_k &\sim \mathcal{N}(0,\sigma_k^2)\\
	\sigma_k^2 &\sim Gamma(s^{\gamma}, r^{\gamma})
\end{align*}

\centering Language Model:
\begin{align*}
\theta_{d} &= LogisticNormal(\mu_d, \Sigma) \\
z_{d,n} &\sim Mult(\theta_d)\\
w_{d,n} &\sim Mult(\beta^{k=z_d,n}_d, r^{\gamma})
\end{align*}

\centering Topical Content:
\begin{align*}
\beta_{d,v}^k &\propto exp(m_v+\kappa_v^{.,k}+\kappa_v^{y,.}+\kappa_v^{y,k}) \\
\kappa_v^{y,k} &\sim Laplace(0,\tau_v^{y,k})\\
\tau_v^{y,k} &\sim Gamma(s^{\kappa}, r^{\kappa})
\end{align*}


\begin{equation*}
\theta_i \stackrel{ind}{\sim} Beta(\alpha, \beta).
\end{equation*}

	\begin{align*}
	 \alpha &=\eta*\mu \\ 
	\beta &= \eta*(1-\mu) 
%	\mu &\sim Beta(1,1) \\
%	\eta & \sim LogNormal(0,1)
	\end{align*}

	\begin{align*}
%	p(\mu, \eta) & \propto p() \\
	p(\theta_i) & \propto p(\mu)p(\eta) \\
	p(\mu) & \propto Beta(1,1) \\
	p(\eta) & \propto LNorm(0,1).
	\end{align*}



\begin{equation*}
p(\theta_1...\theta_{51},\mu,\eta|\mathbf{y}) \propto \prod_{n=1}^{51}\Big[p(y_i|\theta_i)p(\theta_i|\mu,\eta)\Big] p(\mu)p(\eta).
\end{equation*}

%------------------------------------------------

\section{Results}





%\subsubsection{Heading on level 3 (subsubsection)}



%\paragraph{Heading on level 4 (paragraph)}




%\section{Lists}

%------------------------------------------------

%\subsection{Example of list (3*itemize)}
%\begin{itemize}
%	\item First item in a list 
%		\begin{itemize}
%		\item First item in a list 
%			\begin{itemize}
%			\item First item in a list 
%			\item Second item in a list 
%			\end{itemize}
%		\item Second item in a list 
%		\end{itemize}
%	\item Second item in a list 
%\end{itemize}

%------------------------------------------------

%\subsection{Example of list (enumerate)}
%\begin{enumerate}
%\item First item in a list 
%\item Second item in a list 
%\item Third item in a list
%\end{enumerate}

%----------------------------------------------------------------------------------------

\end{document}


